% Options for packages loaded elsewhere
\PassOptionsToPackage{unicode}{hyperref}
\PassOptionsToPackage{hyphens}{url}
%
\documentclass[
]{article}
\usepackage{amsmath,amssymb}
\usepackage{lmodern}
\usepackage{iftex}
\ifPDFTeX
  \usepackage[T1]{fontenc}
  \usepackage[utf8]{inputenc}
  \usepackage{textcomp} % provide euro and other symbols
\else % if luatex or xetex
  \usepackage{unicode-math}
  \defaultfontfeatures{Scale=MatchLowercase}
  \defaultfontfeatures[\rmfamily]{Ligatures=TeX,Scale=1}
\fi
% Use upquote if available, for straight quotes in verbatim environments
\IfFileExists{upquote.sty}{\usepackage{upquote}}{}
\IfFileExists{microtype.sty}{% use microtype if available
  \usepackage[]{microtype}
  \UseMicrotypeSet[protrusion]{basicmath} % disable protrusion for tt fonts
}{}
\makeatletter
\@ifundefined{KOMAClassName}{% if non-KOMA class
  \IfFileExists{parskip.sty}{%
    \usepackage{parskip}
  }{% else
    \setlength{\parindent}{0pt}
    \setlength{\parskip}{6pt plus 2pt minus 1pt}}
}{% if KOMA class
  \KOMAoptions{parskip=half}}
\makeatother
\usepackage{xcolor}
\usepackage[margin=1in]{geometry}
\usepackage{color}
\usepackage{fancyvrb}
\newcommand{\VerbBar}{|}
\newcommand{\VERB}{\Verb[commandchars=\\\{\}]}
\DefineVerbatimEnvironment{Highlighting}{Verbatim}{commandchars=\\\{\}}
% Add ',fontsize=\small' for more characters per line
\usepackage{framed}
\definecolor{shadecolor}{RGB}{248,248,248}
\newenvironment{Shaded}{\begin{snugshade}}{\end{snugshade}}
\newcommand{\AlertTok}[1]{\textcolor[rgb]{0.94,0.16,0.16}{#1}}
\newcommand{\AnnotationTok}[1]{\textcolor[rgb]{0.56,0.35,0.01}{\textbf{\textit{#1}}}}
\newcommand{\AttributeTok}[1]{\textcolor[rgb]{0.77,0.63,0.00}{#1}}
\newcommand{\BaseNTok}[1]{\textcolor[rgb]{0.00,0.00,0.81}{#1}}
\newcommand{\BuiltInTok}[1]{#1}
\newcommand{\CharTok}[1]{\textcolor[rgb]{0.31,0.60,0.02}{#1}}
\newcommand{\CommentTok}[1]{\textcolor[rgb]{0.56,0.35,0.01}{\textit{#1}}}
\newcommand{\CommentVarTok}[1]{\textcolor[rgb]{0.56,0.35,0.01}{\textbf{\textit{#1}}}}
\newcommand{\ConstantTok}[1]{\textcolor[rgb]{0.00,0.00,0.00}{#1}}
\newcommand{\ControlFlowTok}[1]{\textcolor[rgb]{0.13,0.29,0.53}{\textbf{#1}}}
\newcommand{\DataTypeTok}[1]{\textcolor[rgb]{0.13,0.29,0.53}{#1}}
\newcommand{\DecValTok}[1]{\textcolor[rgb]{0.00,0.00,0.81}{#1}}
\newcommand{\DocumentationTok}[1]{\textcolor[rgb]{0.56,0.35,0.01}{\textbf{\textit{#1}}}}
\newcommand{\ErrorTok}[1]{\textcolor[rgb]{0.64,0.00,0.00}{\textbf{#1}}}
\newcommand{\ExtensionTok}[1]{#1}
\newcommand{\FloatTok}[1]{\textcolor[rgb]{0.00,0.00,0.81}{#1}}
\newcommand{\FunctionTok}[1]{\textcolor[rgb]{0.00,0.00,0.00}{#1}}
\newcommand{\ImportTok}[1]{#1}
\newcommand{\InformationTok}[1]{\textcolor[rgb]{0.56,0.35,0.01}{\textbf{\textit{#1}}}}
\newcommand{\KeywordTok}[1]{\textcolor[rgb]{0.13,0.29,0.53}{\textbf{#1}}}
\newcommand{\NormalTok}[1]{#1}
\newcommand{\OperatorTok}[1]{\textcolor[rgb]{0.81,0.36,0.00}{\textbf{#1}}}
\newcommand{\OtherTok}[1]{\textcolor[rgb]{0.56,0.35,0.01}{#1}}
\newcommand{\PreprocessorTok}[1]{\textcolor[rgb]{0.56,0.35,0.01}{\textit{#1}}}
\newcommand{\RegionMarkerTok}[1]{#1}
\newcommand{\SpecialCharTok}[1]{\textcolor[rgb]{0.00,0.00,0.00}{#1}}
\newcommand{\SpecialStringTok}[1]{\textcolor[rgb]{0.31,0.60,0.02}{#1}}
\newcommand{\StringTok}[1]{\textcolor[rgb]{0.31,0.60,0.02}{#1}}
\newcommand{\VariableTok}[1]{\textcolor[rgb]{0.00,0.00,0.00}{#1}}
\newcommand{\VerbatimStringTok}[1]{\textcolor[rgb]{0.31,0.60,0.02}{#1}}
\newcommand{\WarningTok}[1]{\textcolor[rgb]{0.56,0.35,0.01}{\textbf{\textit{#1}}}}
\usepackage{graphicx}
\makeatletter
\def\maxwidth{\ifdim\Gin@nat@width>\linewidth\linewidth\else\Gin@nat@width\fi}
\def\maxheight{\ifdim\Gin@nat@height>\textheight\textheight\else\Gin@nat@height\fi}
\makeatother
% Scale images if necessary, so that they will not overflow the page
% margins by default, and it is still possible to overwrite the defaults
% using explicit options in \includegraphics[width, height, ...]{}
\setkeys{Gin}{width=\maxwidth,height=\maxheight,keepaspectratio}
% Set default figure placement to htbp
\makeatletter
\def\fps@figure{htbp}
\makeatother
\setlength{\emergencystretch}{3em} % prevent overfull lines
\providecommand{\tightlist}{%
  \setlength{\itemsep}{0pt}\setlength{\parskip}{0pt}}
\setcounter{secnumdepth}{-\maxdimen} % remove section numbering
\usepackage{booktabs}
\usepackage{longtable}
\usepackage{array}
\usepackage{multirow}
\usepackage{wrapfig}
\usepackage{float}
\usepackage{colortbl}
\usepackage{pdflscape}
\usepackage{tabu}
\usepackage{threeparttable}
\usepackage{threeparttablex}
\usepackage[normalem]{ulem}
\usepackage{makecell}
\usepackage{xcolor}
\ifLuaTeX
  \usepackage{selnolig}  % disable illegal ligatures
\fi
\IfFileExists{bookmark.sty}{\usepackage{bookmark}}{\usepackage{hyperref}}
\IfFileExists{xurl.sty}{\usepackage{xurl}}{} % add URL line breaks if available
\urlstyle{same} % disable monospaced font for URLs
\hypersetup{
  pdftitle={My Reproducible Report},
  pdfauthor={A forward thinking researcher},
  hidelinks,
  pdfcreator={LaTeX via pandoc}}

\title{My Reproducible Report}
\author{A forward thinking researcher}
\date{2023-01-17}

\begin{document}
\maketitle

\hypertarget{overview}{%
\subsection{Overview}\label{overview}}

This file provides a quick look at how each component of an R Markdown
file works together to produce an integrated report.

Remember, R Markdown allows us to:

\begin{itemize}
\tightlist
\item
  \textbf{Combine narriative text, code, and output} into a single
  document
\item
  Incorporate mathematical expressions or equations
\item
  And render/export the .Rmd file to a variety of output file formats!
\item
  The YAML header specified in this \texttt{.Rmd} file enables us to
  render the output to:

  \begin{itemize}
  \tightlist
  \item
    A Microsoft Word document (\texttt{.docx})
  \item
    A \texttt{pdf} file
  \item
    Or \texttt{html} file
  \end{itemize}
\end{itemize}

\hypertarget{load-packages}{%
\subsection{Load packages}\label{load-packages}}

In this code chunk, we are loading some relevant package libraries. By
using the \texttt{include\ =\ FALSE} in our chunk options, we are
telling the \texttt{.Rmd} file to exclude the code/output from the
exported document.

\hypertarget{create-a-plot-in-the-below-code-chunk}{%
\subsection{Create a plot in the below code
chunk}\label{create-a-plot-in-the-below-code-chunk}}

Below we create a faceted figure using the gapminder dataset. We use the
chunk options to specify the figure output height and width, as well as
to create a figure caption. Notice that we can also use markdown syntax
to format the caption text (e.g., \textbf{bold face text})

\begin{Shaded}
\begin{Highlighting}[]
\CommentTok{\# Create life expectancy by year figure}
\FunctionTok{ggplot}\NormalTok{(data, }\FunctionTok{aes}\NormalTok{(}\AttributeTok{x =}\NormalTok{ year, }\AttributeTok{y =}\NormalTok{ lifeExp, }\AttributeTok{color =}\NormalTok{ country)) }\SpecialCharTok{+}
  \FunctionTok{geom\_line}\NormalTok{(}\AttributeTok{show.legend =} \ConstantTok{FALSE}\NormalTok{) }\SpecialCharTok{+}
  \FunctionTok{scale\_x\_continuous}\NormalTok{(}
    \AttributeTok{expand =} \FunctionTok{c}\NormalTok{(}\DecValTok{0}\NormalTok{, }\DecValTok{0}\NormalTok{),}
    \AttributeTok{limits =} \FunctionTok{c}\NormalTok{(}\DecValTok{1950}\NormalTok{, }\DecValTok{2010}\NormalTok{),}
    \AttributeTok{breaks =} \FunctionTok{seq}\NormalTok{(}\DecValTok{1950}\NormalTok{, }\DecValTok{2010}\NormalTok{, }\DecValTok{10}\NormalTok{)}
\NormalTok{  ) }\SpecialCharTok{+}
  \FunctionTok{scale\_y\_continuous}\NormalTok{(}
    \AttributeTok{expand =} \FunctionTok{c}\NormalTok{(}\DecValTok{0}\NormalTok{, }\DecValTok{0}\NormalTok{),}
    \AttributeTok{limits =} \FunctionTok{c}\NormalTok{(}\DecValTok{20}\NormalTok{, }\DecValTok{90}\NormalTok{),}
    \AttributeTok{breaks =} \FunctionTok{seq}\NormalTok{(}\DecValTok{20}\NormalTok{, }\DecValTok{90}\NormalTok{, }\DecValTok{10}\NormalTok{)}
\NormalTok{  ) }\SpecialCharTok{+}
  \FunctionTok{labs}\NormalTok{(}\AttributeTok{x =} \StringTok{"Year"}\NormalTok{, }\AttributeTok{y =} \StringTok{"Life Expectancy at Birth (Years)"}\NormalTok{) }\SpecialCharTok{+}
  \FunctionTok{facet\_wrap}\NormalTok{(}\SpecialCharTok{\textasciitilde{}}\NormalTok{ continent) }\SpecialCharTok{+}
  \FunctionTok{theme\_bw}\NormalTok{() }\SpecialCharTok{+}
  \FunctionTok{theme}\NormalTok{(}
    \AttributeTok{axis.text =} \FunctionTok{element\_text}\NormalTok{(}\AttributeTok{face =} \StringTok{"bold"}\NormalTok{, }\AttributeTok{color =} \StringTok{"black"}\NormalTok{),}
    \AttributeTok{axis.text.x =} \FunctionTok{element\_text}\NormalTok{(}\AttributeTok{angle =} \DecValTok{45}\NormalTok{, }\AttributeTok{vjust =} \FloatTok{0.6}\NormalTok{),}
    \AttributeTok{axis.title =} \FunctionTok{element\_text}\NormalTok{(}\AttributeTok{face =} \StringTok{"bold"}\NormalTok{),}
    \AttributeTok{strip.text =} \FunctionTok{element\_text}\NormalTok{(}\AttributeTok{face =} \StringTok{"bold"}\NormalTok{),}
    \AttributeTok{panel.spacing.x =} \FunctionTok{unit}\NormalTok{(}\FloatTok{1.5}\NormalTok{, }\StringTok{"lines"}\NormalTok{),}
    \AttributeTok{panel.spacing.y =} \FunctionTok{unit}\NormalTok{(}\FloatTok{1.5}\NormalTok{, }\StringTok{"lines"}\NormalTok{)}
\NormalTok{  )}
\end{Highlighting}
\end{Shaded}

\begin{figure}

{\centering \includegraphics{my_reproducible_report_files/figure-latex/unnamed-chunk-2-1} 

}

\caption{**Life expecetancy over time by country, stratified by continent.**}\label{fig:unnamed-chunk-2}
\end{figure}

\hypertarget{how-to-embed-a-simple-table-using-the-kableextra-package}{%
\subsection{How to embed a simple table (using the KableExtra
package)}\label{how-to-embed-a-simple-table-using-the-kableextra-package}}

Here's a walkthrough of what each line of code is doing:

\begin{itemize}
\tightlist
\item
  Reshape the \texttt{gapminder} dataset so it's in a long-data format
\item
  Filter for only life expectancy values obtained for the year 2007
\item
  Apply a group structure to the data frame (to compute grouped summary
  statistics)
\item
  Compute some simple summary statistics, by continent
\item
  Round to two decimal places across numeric columns
\item
  Use the Kable package to generate a simple, but nicely formatted table
\end{itemize}

\begin{Shaded}
\begin{Highlighting}[]
\NormalTok{data }\SpecialCharTok{\%\textgreater{}\%}
  \FunctionTok{pivot\_longer}\NormalTok{(lifeExp}\SpecialCharTok{:}\NormalTok{gdpPercap, }\AttributeTok{names\_to =} \StringTok{"parameters"}\NormalTok{, }\AttributeTok{values\_to =} \StringTok{"values"}\NormalTok{) }\SpecialCharTok{\%\textgreater{}\%}
  \FunctionTok{filter}\NormalTok{(year }\SpecialCharTok{==} \DecValTok{2007} \SpecialCharTok{\&}\NormalTok{ parameters }\SpecialCharTok{==} \StringTok{"lifeExp"}\NormalTok{) }\SpecialCharTok{\%\textgreater{}\%}
  \FunctionTok{group\_by}\NormalTok{(continent) }\SpecialCharTok{\%\textgreater{}\%}
  \FunctionTok{summarize}\NormalTok{(}\AttributeTok{Mean =} \FunctionTok{mean}\NormalTok{(values, }\AttributeTok{na.rm =} \ConstantTok{TRUE}\NormalTok{),}
            \StringTok{\textasciigrave{}}\AttributeTok{Std. Dev}\StringTok{\textasciigrave{}} \OtherTok{=} \FunctionTok{sd}\NormalTok{(values, }\AttributeTok{na.rm =} \ConstantTok{TRUE}\NormalTok{),}
            \StringTok{\textasciigrave{}}\AttributeTok{Min.}\StringTok{\textasciigrave{}}\OtherTok{=} \FunctionTok{min}\NormalTok{(values, }\AttributeTok{na.rm =} \ConstantTok{TRUE}\NormalTok{),}
            \StringTok{\textasciigrave{}}\AttributeTok{Max.}\StringTok{\textasciigrave{}} \OtherTok{=} \FunctionTok{max}\NormalTok{(values, }\AttributeTok{na.rm =} \ConstantTok{TRUE}\NormalTok{)) }\SpecialCharTok{\%\textgreater{}\%}
  \FunctionTok{mutate}\NormalTok{(}\FunctionTok{across}\NormalTok{(is.numeric, }\SpecialCharTok{\textasciitilde{}} \FunctionTok{round}\NormalTok{(., }\DecValTok{2}\NormalTok{))) }\SpecialCharTok{\%\textgreater{}\%}
  \FunctionTok{kbl}\NormalTok{() }\SpecialCharTok{\%\textgreater{}\%}
  \FunctionTok{kable\_classic}\NormalTok{()}
\end{Highlighting}
\end{Shaded}

\begin{table}
\centering
\begin{tabular}[t]{l|r|r|r|r}
\hline
continent & Mean & Std. Dev & Min. & Max.\\
\hline
Africa & 54.81 & 9.63 & 39.61 & 76.44\\
\hline
Americas & 73.61 & 4.44 & 60.92 & 80.65\\
\hline
Asia & 70.73 & 7.96 & 43.83 & 82.60\\
\hline
Europe & 77.65 & 2.98 & 71.78 & 81.76\\
\hline
Oceania & 80.72 & 0.73 & 80.20 & 81.24\\
\hline
\end{tabular}
\end{table}

\hypertarget{some-examples-of-how-you-can-embed-quantitative-statments-into-your-file-with-latex.}{%
\subsubsection{Some examples of how you can embed quantitative statments
into your file with
LaTeX.}\label{some-examples-of-how-you-can-embed-quantitative-statments-into-your-file-with-latex.}}

\begin{itemize}
\tightlist
\item
  Enclosing the expression in pairs of two dollar signs
  (\texttt{\$\$}..\texttt{\$\$}) generates a stand-alone equation. \[
  \begin{equation}
  \hat{Y}_i = \hat{\beta}_0 + \hat{\beta}_1 X_i + \hat{\epsilon}_i
  \end{equation}
  \]
\item
  In contrast, wrapping the expression in a pair of single dollar signs
  (\texttt{\$}..\texttt{\$}) creates an in-line quantitative expression.
\end{itemize}

\begin{itemize}
\tightlist
\item
  This is some text with an in-line equation like
  \(\hat{Y}_i = \hat{\beta}_0 + \hat{\beta}_1 X_i + \hat{\epsilon}_i\).
  Pretty cool, right?
\end{itemize}

\end{document}
